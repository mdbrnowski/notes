\appendix
\section{Graf redukcji między problemami}
\label{app:graph}

Kolorem czerwonym zaznaczono redukcje, które są oczywiste --- jeden problem jest szczególnym przypadkiem drugiego.

\begin{tikzpicture}[
    node distance=1.5cm,
    problem/.style={
        rectangle, draw, minimum width=1.3cm, minimum height=.5cm, align=center,
        font=\scriptsize
    },
    label/.style={
        midway, fill=white, inner sep=2pt,
        font=\scriptsize
    },
    reduction/.style={->, thick},
    bireduction/.style={<->, thick},
    obvious/.style={->, thick, AccColor1, densely dashed},
]
    \node[problem] (SAT) {SAT};

    \node[problem, below=of SAT] (ThreeSAT) {3-SAT};
    \draw[reduction] (SAT) -- node[label] {\ref{t:3-SAT}} (ThreeSAT);
    \draw[obvious] (ThreeSAT) to[bend right] (SAT);

    \node[problem, below left=0.8cm and 3.3cm of ThreeSAT] (IS) {\textsc{Independent Set}};
    \draw[reduction] (ThreeSAT) -- node[label] {\ref{t:Independent Set}} (IS);

    \node[problem, below left=2.8cm and -0.8cm of IS] (Clique) {\textsc{Clique}};
    \draw[bireduction] (IS) -- node[label] {\ref{t:Clique}} (Clique);

    \node[problem, below right=2.8cm and -1.3cm of IS] (VertexCover) {\textsc{Vertex Cover}};
    \draw[bireduction] (Clique) -- (VertexCover);
    \draw[bireduction] (IS) -- node[label] {\ref{t:Vertex Cover}} (VertexCover);

    \node[problem, below left=of ThreeSAT] (ThreeColor) {\textsc{$3$-Color}};
    \draw[reduction] (ThreeSAT) -- node[label] {\ref{t:3-Color}} (ThreeColor);

    \node[problem, below=of ThreeColor] (Planar3Color) {\textsc{Planar $3$-Color}};
    \draw[reduction] (ThreeColor) -- (Planar3Color);
    \draw[obvious] (Planar3Color) to[bend right] (ThreeColor);

    \node[problem, below=of ThreeSAT] (ThreeThreeSAT) {$(3,3)$-SAT};
    \draw[reduction] (ThreeSAT) -- node[label] {\ref{t:3,3-SAT}} (ThreeThreeSAT);
    \draw[obvious] (ThreeThreeSAT) to[bend right] (ThreeSAT);

    \node[problem, below right=of ThreeSAT] (SetCover) {\textsc{Set Cover}};
    \draw[reduction] (ThreeSAT) -- node[label] {\ref{t:Set Cover}} (SetCover);
    \draw[obvious] (VertexCover) to[bend right] (SetCover);

    \node[problem, below=of SetCover] (X3C) {X3C};
    \draw[reduction] (ThreeThreeSAT) -- node[label] {\ref{t:X3C}} (X3C);
    \draw[obvious] (X3C) to[bend right] (SetCover);

    \node[problem, below left=2cm and 0cm of X3C] (SubsetSum) {\textsc{Subset Sum}};
    \draw[reduction] (X3C) -- node[label] {\ref{t:Subset Sum}} (SubsetSum);

    \node[problem, below right=2cm and 0cm of X3C] (RX3C) {RX3C};
    \draw[reduction] (X3C) -- node[label] {\ref{r:RX3C}} (RX3C);
    \draw[obvious] (RX3C) to[bend right] (X3C);

    \node[problem, below=of SubsetSum] (Partition) {\textsc{Partition}};
    \draw[reduction] (SubsetSum) -- (Partition);
    \draw[obvious] (Partition) to[bend right] (SubsetSum);

    \node[problem, below right=0.8cm and 3.6 cm of ThreeSAT] (DHamPath) {\textsc{Dir. Hamiltonian Path}};
    \draw[reduction] (ThreeSAT) -- node[label] {\ref{t:Directed Hamiltonian Path}} (DHamPath);

    \node[problem, below=of DHamPath] (HamPath) {\textsc{Hamiltonian Path}};
    \draw[reduction] (DHamPath) -- node[label] {\ref{t:Hamiltonian Path}} (HamPath);
    \draw[obvious] (HamPath) to[bend right] (DHamPath);
\end{tikzpicture}
