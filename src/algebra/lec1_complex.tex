\begin{definition}
    Liczba zespolona $z$ to uporządkowana para liczb rzeczywistych. Pierwszy element tej pary to \vocab{część rzeczywista}, oznaczana symbolem $\Re(z)$, a drugi to \vocab{część urojona}, oznaczana symbolem $\Im(z)$. Zbiór liczb zespolonych oznaczamy przez $\CC$.
\end{definition}

Liczby zespolone można reprezentować w kilku postaciach, jedna z nich to \vocab{postać algebraiczna}. Używając jej, liczba $z = (x, y)$ jest zapisywana jako
\[ z = x + iy, \]
gdzie $i$ nazywamy \vocab{jednostką urojoną}, która spełnia
\[ i^2 = -1. \]

Niech $z_1 = x_1 + iy_1$ oraz $z_2 = x_2 + iy_2$. Określamy:
\begin{itemize}
    \item dodawanie $z_1 + z_2 = x_1 + x_2 + i(y_1 + y_2)$,
    \item mnożenie $\begin{aligned}[t] z_1z_2 &= x_1x_2 + ix_1y_2 + ix_2y_1 + i^2y_1y_2 \\ &= x_1x_2 - y_1y_2 + i(x_1y_2 + x_2y_1).\end{aligned}$
\end{itemize}

\begin{corollary}
    Dodawanie i mnożenie liczb zespolonych jest przemienne i łączne. Mnożenie jest rozdzielne względem dodawania.
\end{corollary}

\begin{definition}
    Sprzężenie liczby zespolonej $z = x + iy$ to liczba $\ol{z} = x - iy$.
\end{definition}

\begin{definition}
    \label{d:magnitude}
    Moduł liczby zespolonej $z = x + iy$ to liczba $|z| = \sqrt{x^2 + y^2}$.
\end{definition}

Zachodzi pewna własność, wynikająca ze wzoru skróconego mnożenia:
\[ z\ol{z} = (x + iy)(x - iy) = x^2 - i^2y^2 = x^2 + y^2 \]
\begin{equation}
    z\ol{z} = |z|^2
\end{equation}

Powyższa liczba jest liczbą rzeczywistą, więc znaleźliśmy prosty sposób na dzielenie liczb zespolonych przez siebie, mnożąc licznik i mianownik przez sprzężenie mianownika. Na przykład:
\[ \frac{1 + 2i}{-1 - i} = \frac{(1 + 2i)(-1 + i)}{(-1 - i)(-1 + i)} = \frac{-3 -i}{2} = \frac{-3}{2} - \frac{i}{2}. \]

\begin{lemma}
    Oprócz $z\ol{z} = |z|^2$, zachodzą również równości:
    \begin{itemize}
        \item $|\ol{z}| = |z|$
        \item $\ol{z_1 + z_2} = \ol{z_1} + \ol{z_2}$
        \item $\ol{z_1z_2} = \ol{z_1}\cdot\ol{z_2}$
        \item $|z_1z_2| = |z_1||z_2|$
    \end{itemize}
\end{lemma}
Ich dowody można w łatwy sposób przeprowadzić z definicji poszczególnych działań.

\subsection{Interpretacja geometryczna liczb zespolonych}
Liczby zespolone można interpretować jako punkty na \vocab{płaszczyźnie zespolonej}. Dla przykładu liczba $z = 3 + 2i$.

\begin{center}
    \begin{tikzpicture}
        \tkzInit[xmin=-.7, xmax=3.7, ymin=-.7, ymax=2.7]
        \tkzDefPoints{0/0/O,3/2/z}
        \tkzGrid
        \tkzDrawX[label=$\Re$,thick] \tkzDrawY[label=$\Im$,thick]
        \tkzDrawSegments(O,z)
        \tkzDrawPoints(z)
        \tkzLabelPoints[above right](z)
    \end{tikzpicture}
\end{center}

\begin{fact}
    Moduł liczby zespolonej $z$ to długość wektora wodzącego tej liczby na płaszczyźnie zespolonej.
\end{fact}
\begin{proof}
    Wynika to z twierdzenia Pitagorasa oraz definicji modułu (\ref{d:magnitude}).
\end{proof}

Możemy wyprowadzić \vocab{postać trygonometryczną} liczby zespolonej, która będzie operować na długości wektora wodzącego oraz kącie skierowanym. Mamy więc
\[ z = |z|(\cos\varphi + i\sin\varphi) \]
gdzie $\varphi$ to miara kąta skierowanego między wektorem wodzącym liczby zespolonej $z$ a~osią liczb rzeczywistych. Ten kąt nazywany jest \vocab{argumentem} i oznaczany przez $\Arg(z)$. Argument nie jest określony jednoznacznie -- dowolne dwa argumenty jednej liczby różnią się o wielokrotność $2\pi$. Jeśli argument jest w przedziale $[0, 2\pi)$, to mówimy, że jest to \vocab{argument główny} liczby $z$ i oznaczamy $\arg(z)$.

Za pomocą podstawowej trygonometrii możemy łatwo zamieniać postać algebraiczną i trygonometryczną między sobą.

\begin{center}
    \begin{tikzpicture}
        \tkzInit[xmin=-.7, xmax=3.7, ymin=-.7, ymax=2.7]
        \tkzDefPoints{0/0/O,1/0/A,3/2/z}
        \tkzDefPointBy[projection=onto O--A](z) \tkzGetPoint{z'}
        \tkzGrid
        \tkzDrawX[label=$\Re$,thick] \tkzDrawY[label=$\Im$,thick]
        \tkzDrawSegment[dim={$|z|$,2mm,}, dim style/.style={sloped,dashed}](O,z)
        \tkzDrawSegment[dim={$|z|\cos\varphi$,-2mm,}, dim style/.style={sloped,dashed}](O,z')
        \tkzDrawSegment[dim={$|z|\sin\varphi$,2mm,}, dim style/.style={sloped,dashed}](z,z')
        \tkzMarkAngle[size=1.1](z',O,z)
        \tkzLabelAngle[pos=.8](z',O,z){$\varphi$}
        \tkzDrawPoints(z)
        \tkzLabelPoints[above right](z)
    \end{tikzpicture}
\end{center}

\begin{equation}
    \Re{z} = |z|\cos\varphi, \hspace{2em} \Im{z} = |z|\sin\varphi
\end{equation}

Na potrzeby dalszych rozważań przyjmujemy, że $\arg(0) = 0$.

\begin{fact}
    Odległość między liczbami $z_1$ i $z_2$ na płaszczyźnie zespolonej wynosi $|z_1 - z_2|$.
\end{fact}

\begin{lemma}
    Zachodzą następujące nierówności:
    \begin{itemize}
        \item $|z_1 + z_2| \leq |z_1| + |z_2|$
        \item $||z_1| - |z_2|| \leq |z_1 - z_2|$
    \end{itemize}
\end{lemma}

Możemy łatwo mnożyć dwie liczby zespolone w postaci trygonometrycznej przez siebie za pomocą poniższego wzoru.
\begin{equation}
    \label{eq:complex_prod}
    \begin{aligned}
        z_1 \cdot z_2 &= |z_1|(\cos\varphi_1 + i\sin\varphi_1)|z_2|(\cos\varphi_2 + i\sin\varphi_2) \\
                      &= |z_1||z_2|(\cos\varphi_1\cos\varphi_2 - \sin\varphi_1\sin\varphi_2 + i(\cos\varphi_1\sin\varphi_2 + \sin\varphi_1\cos\varphi_2)) \\
                      &= |z_1||z_2|(\cos(\varphi_1 + \varphi_2) + i\sin(\varphi_1 + \varphi_2))
    \end{aligned}
\end{equation}

Stosując wzór \ref{eq:complex_prod} $n$ razy otrzymujemy dowód następującego twierdzenia.

\begin{theorem}[wzór de Moivre'a]
    \label{t:demoivre}
    Dla $z = |z|(\cos\varphi + i\sin\varphi)$ oraz $n \in \ZZ$ zachodzi równość
    \[ z^n = |z|^n(\cos n\varphi + i\sin n\varphi) \]
\end{theorem}

Wzór de Moivre'a zapewnia prosty sposób na potęgowanie liczb zespolonych. Dlatego, mając za zadanie obliczyć
\[ (-2\sqrt{3} - 2i)^{16} \]
najłatwiej będzie zmienić postać liczby do postaci trygonometrycznej, a następnie skorzystać ze wzoru de Moivre'a.

\begin{definition}[pierwiastek liczby zespolonej]
    Jeśli $z$ jest liczbą zespoloną, to $\sqrt[n]{z}$ jest zbiorem wszystkich takich $w \in \CC$, że $w^n = z$.
\end{definition}

Korzystając ze wzoru de Moivre'a (twierdzenie \ref{t:demoivre}), łatwo wyprowadzić wzór
\begin{equation}
    \label{eq:complex_root}
    \sqrt[n]{z} = \sqrt[n]{|z|}\left(\cos\frac{\varphi + 2k\pi}{n} + i\sin\frac{\varphi + 2k\pi}{n}\right), k \in \ZZ
\end{equation}

\begin{fact}
    Pierwiastków $n$-tego stopnia z $z \neq 0$ jest dokładnie $n$ i leżą one w równych odstępach na okręgu o środku w $0$ i promieniu $\sqrt[n]{|z|}$.
\end{fact}
\begin{proof}
    Dla $k \in \{0, 1, \ldots, n-1\}$ liczba z równości \ref{eq:complex_root} będzie przyjmować różne wartości (wynika to z okresowości funkcji trygonometrycznych). Liczby te będą na wspomnianym okręgu (to wynika wprost z postaci trygonometrycznej), a ich argumenty główne różnić będzie wielokrotność $\frac{2\pi}{n}$.
\end{proof}

\subsection{Postać wykładnicza}
Postać $z = |z|e^{i\varphi}$ liczby zespolonej będziemy nazywać \vocab{postacią wykładniczą} tej liczby.

\begin{theorem}[wzór Eulera]
    Dla każdego $\varphi \in \RR$ zachodzi
    \[ e^{i\varphi} = \cos\varphi + i\sin\varphi. \]
\end{theorem}
\begin{proof}
    Standardowo dowodzi się wzoru Eulera za pomocą szeregów Taylora. Pokażemy inny, mniej oczywisty, ale bardziej elementarny dowód.

    Niech $f(\varphi) = e^{-i\varphi}\left(\cos\varphi + i\sin\varphi\right)$. Zróżniczkujmy:
    \begin{align*}
        f'(\varphi) &= -ie^{-i\varphi}\left(\cos\varphi + i\sin\varphi\right) + e^{-i\varphi}\left(-\sin\varphi + i\cos\varphi\right) = \\
        &= e^{-i\varphi}\left(-i\cos\varphi + i\cos\varphi + \sin\varphi - \sin\varphi\right) = 0.
    \end{align*}
    Z tego wynika, że funkcja $f$ jest stała, więc
    \[ f(\varphi) \equiv f(0) = 1, \]
    \[ \therefore e^{i\varphi} = \cos\varphi + i\sin\varphi. \]
\end{proof}