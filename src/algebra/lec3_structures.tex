\vocab{Działaniem} (wewnętrznym) w zbiorze $A$ nazwiemy każde odwzorowanie $h$ takie, że
$$ h : A \times A \to A. $$
\vocab{Działaniem zewnętrznym} w zbiorze $A$ jest odwzorowanie
$$ h : F \times A \to A. $$

Jeśli zamiast $h$ weźmiemy jakiś symbol, na przykład $\circ$, to, zamiast $h(a, b)$ będziemy pisać $a \circ b$.

\begin{definition}[rodzaje działań]
    W zbiorze z działaniem $(A, \circ)$ działanie $\circ$ jest:
    \begin{itemize}
        \item \vocab{łączne} $\iff \forall x, y, z \in A : (x \circ y) \circ z = x \circ (y \circ z)$,
        \item \vocab{przemienne} $\iff \forall x, y, \in A : x \circ y = y \circ x$.
    \end{itemize}
\end{definition}

Jeśli dla pewnego elementu $e \in A$ zachodzi
$$ \forall x \in A : x \circ e = e \circ x = x, $$
to $e$ jest \vocab{elementem neutralnym}.

\begin{fact}
    Jeżeli w zbiorze $A$ z działaniem $\circ$ istnieje element neutralny, to jest on jedyny.
\end{fact}
\begin{proof}
    Jeśli mielibyśmy dwa elementy neutralne $e_1, e_2$ to mamy
    $$ e_1 \circ e_2 = e_1 = e_2. $$
\end{proof}

Jeżeli istnieje element neutralny $e \in A$ działania $\circ$, to \vocab{elementem symetrycznym} do $x \in A$ jest taki element $x' \in A$, że
$$ x \circ x' = e = x' \circ x. $$

\begin{lemma}
    Jeśli działanie $\circ$ jest łączne w zbiorze $A$ i istnieje element neutralny $e \in A$, to jeśli dany element $x \in A$ ma element symetryczny, to jest on jedyny oraz zachodzi $(x')' = x$.
\end{lemma}
\begin{proof}
    Jeśli mielibyśmy dwa elementy symetryczne $x'_1, x'_2$, to mamy
    $$ x'_1 = x'_1 \circ e = x'_1 \circ (x \circ x'_2) = (x'_1 \circ x) \circ x'_2 = e \circ x'_2 = x'_2. $$

    Ponadto z definicji elementu symetrycznego mamy
    $$ x' \circ x = e $$
    oraz
    $$ x' \circ (x')' = e, $$
    a więc $x$ jest elementem symetrycznym $x'$, ergo $(x')' = x$.
\end{proof}

\subsection{Grupy}
\begin{definition}
    Grupa to para $(A, \circ)$, gdzie $A$ jest zbiorem, a działanie $\circ$ jest:
    \begin{enumerate}[noitemsep,nolistsep]
        \item wewnętrzne,
        \item łączne,
        \item ma element neutralny,
        \item a każdy element $x \in A$ ma element symetryczny.
    \end{enumerate}
\end{definition}

\begin{definition}
    Grupa abelowa (przemienna) to grupa, w której działanie $\circ$ jest przemienne.
\end{definition}

\begin{example}
    Przykłady grup:
    \begin{enumerate}
        \item $(\ZZ, +)$ -- grupa abelowa,
        \item $(\ZZ_n, +_n)$ -- grupa abelowa\footnote{Symbol $\ZZ_n$ oznacza zbiór $\{0, 1, \ldots, n-1\}$, a $+_n$ operację dodawania modulo $n$.},
        \item $(\QQ_+, \cdot)$ -- grupa abelowa,
        \item grupą nieabelową jest grupa obrotów danego obiektu o $90\dg$ względem dowolnej z trzech osi.
    \end{enumerate}
\end{example}

\begin{theorem}
    \label{t:prime_n->group}
    $(\ZZ_n \setminus \{0\}, \cdot_n)$ jest grupą wtedy i tylko wtedy, gdy $n \geq 2$ jest liczbą pierwszą.
\end{theorem}
Łatwo sprawdzić, że mnożenie modulo $n$ w zbiorze $\ZZ_n \setminus \{0\}$ jest wewnętrzne i łączne. Ma również element neutralny $1$. Będziemy więc dowodzić jedynie istnienia elementu symetrycznego dla każdego elementu.
\begin{proof}[Dowód wystarczalności]\renewcommand{\qedsymbol}{}
    Załóżmy przeciwnie, że istnieje $k \in \ZZ_n \setminus \{0, 1\}$ takie, że $k \mid n$. Skoro $(\ZZ_n \setminus \{0\}, \cdot_n)$ jest grupą, to $k$ ma element symetryczny $k^{-1}$. Zachodzi więc
    $$ kk^{-1} \equiv 1 \pmod{n}, $$
    czyli inaczej
    $$ \exists m \in \ZZ : kk^{-1} - 1 = mn. $$
    Co jednak prowadzi do sprzeczności, ponieważ
    $$ kk^{-1} - 1 \not\equiv mn \pmod{k} $$
    $$ - 1 \not\equiv 0 \pmod{k}. $$
\end{proof}
\begin{proof}[Dowód dostateczności]
    Skoro $n$ jest liczbą pierwszą, to z małego twierdzenia Fermata mamy
    $$ a^{n-1} \equiv 1 \pmod{n} $$
    dla każdego $a \in \ZZ_n \setminus \{0\}$.
    Z tego wynika, że dla dowolnego elementu $a$ jego elementem symetrycznym będzie $a^{n-2}$.
\end{proof}

\subsection{Pierścienie i ciała}
\begin{definition}
    Pierścień to trójka $(P, \circ, *)$, gdzie $P$ jest zbiorem, $\circ, *$ to działania wewnętrzne oraz
    \begin{enumerate}[noitemsep,nolistsep]
        \item $(P, \circ)$ jest grupą abelową
        \item działanie $*$ jest łączne
        \item działanie $*$ jest rozdzielne względem $\circ$, czyli
        $$\forall x, y, z \in P : \begin{aligned}& (x \circ y) * z = (x * z) \circ (y * z), \\
                                                 & x * (y \circ z) = (x * y) \circ (x * z).\end{aligned} $$
    \end{enumerate}
\end{definition}

\begin{definition}
    Pierścień przemienny to pierścień $(P, \circ, *)$, w którym $*$ jest działaniem przemiennym\footnote{Wtedy też rozdzielność prawo- i lewostronna stają się tożsame.}.
\end{definition}

Pierwsze działanie w pierścieniu nazywamy \vocab{działaniem addytywnym} i oznaczamy symbolem $+$. Element neutralny tego działania nazywamy zerem ($\mathbf{0}$), a element symetryczny do elementu $x$ nazywamy elementem przeciwnym i oznaczamy $-x$.

Drugie działanie nazywamy \vocab{działaniem multiplikatywnym} i oznaczamy przez $\cdot$. Jeśli w $P$ dodatkowo istnieje element neutralny tego działania, to ten element nazywamy jedynką ($\mathbf{1}$), a pierścień nazywamy \vocab{pierścieniem z jedynką}. Element symetryczny do elementu $x$ nazywamy elementem odwrotnym i oznaczamy $x^{-1}$.

\begin{definition}
    Dzielnikiem zera jest taki element pierścienia $a \neq \mathbf{0}$, że istnieje niezerowy element $b$, dla którego zachodzi $a \cdot b = \mathbf{0}$.
\end{definition}

\begin{definition}
    Pierścień całkowity to pierścień przemienny z jedynką, w którym nie ma dzielników zera.
\end{definition}

\begin{lemma}
    \label{l:cancellation_property}
    W pierścieniach całkowitych zachodzi \vocab{własność skracania}, to znaczy, że dla elementów pierścienia $a, b, c$ przy $c \neq \mathbf{0}$ zachodzi
    $$ ac = bc \implies a = b. $$
\end{lemma}
\begin{proof}
    Jeśli $ac = bc$, to $ac - bc = \mathbf{0}$. Z rozdzielności dostajemy
    $$ (a - b)c = \mathbf{0}. $$
    W pierścieniu całkowitym nie ma jednak dzielników zera, więc $a - b = \mathbf{0}$, co dowodzi tezy.
\end{proof}

\begin{definition}
    Ciało to pierścień z jedynką, w którym dla każdego elementu $x \neq \mathbf{0}$ istnieje element odwrotny $x^{-1}$.
\end{definition}

\vocab{Ciałem przemiennym} będzie takie ciało, w którym działanie multiplikatywne $\cdot$ jest przemienne\footnote{Większość autorów już w definicji ciała wymaga przemienności (wtedy ciało nazywamy pierścieniem z~dzieleniem). Przyjęło się tak zwłaszcza w literaturze angielskiej (ciało przemienne to \textit{field}, a ciało to \textit{division ring} lub \textit{skew field}) i niemieckiej (odpowiednio \textit{Körper} i \textit{Schiefkörper}). Odwrotnie --- czyli zgodnie z naszą konwencją --- jest w literaturze francuskiej (odpowiednio \textit{corps commutatif} i \textit{corps}) oraz rosyjskiej (\cyrillic{\textit{поле}} [pole] i \cyrillic{\textit{тело}} [tieło]).}.

Można zauważyć, że struktura $(K, +, \cdot)$ jest ciałem (przemiennym), jeżeli:
\begin{enumerate}[noitemsep,nolistsep]
    \item $(K, +)$ jest grupą abelową,
    \item $(K \setminus \{\mathbf{0}\}, \cdot)$ jest grupą (przemienną),
    \item zachodzi warunek rozdzielności $\cdot$ względem $+$.
\end{enumerate}

\begin{lemma}
    \label{l:a0=0}
    Dla każdego elementu ciała $a$ zachodzi $a \cdot \mathbf{0} = \mathbf{0}$.
\end{lemma}
\begin{proof}
    \begin{align*}             a \cdot \mathbf{0} &= a \cdot (\mathbf{0} + \mathbf{0}) \\
                               a \cdot \mathbf{0} &= a \cdot \mathbf{0} + a \cdot \mathbf{0} \\
        a \cdot \mathbf{0} + - a \cdot \mathbf{0} &= a \cdot \mathbf{0} + a \cdot \mathbf{0} + - a \cdot \mathbf{0} \\
                                       \mathbf{0} &= a \cdot \mathbf{0} + \mathbf{0} \\
                                       \mathbf{0} &= a \cdot \mathbf{0}
    \end{align*}
\end{proof}

\begin{theorem}
    Każde ciało przemienne jest pierścieniem całkowitym.
\end{theorem}
\begin{proof}
    Załóżmy przeciwnie, że istnieją dzielniki zera, czyli takie dwa elementy ciała $x, y$, że $x, y \neq \mathbf{0}$ oraz $x \cdot y = \mathbf{0}$. Mamy
    \begin{align*}
        x \cdot y &= \mathbf{0} \\
        x^{-1} \cdot x \cdot y &= x^{-1} \cdot \mathbf{0} \\
        y &= x^{-1} \cdot \mathbf{0},
    \end{align*}
    co, na mocy lematu \ref{l:a0=0}, jest sprzecznością z założeniem.
\end{proof}

\begin{theorem}
    Każdy skończony pierścień całkowity jest ciałem przemiennym.
\end{theorem}
\begin{proof}
    Załóżmy przeciwnie, że istnieje element pierścienia $a \neq \mathbf{0}$, który nie ma elementu odwrotnego. Rozważmy iloczyny $aa_1, aa_2, aa_3, \ldots$ elementu $a$ ze wszystkimi innymi elementami pierścienia (w tym z $\mathbf{1}$). Z założenia nie ma wśród nich jedynki, więc, skoro $\cdot$ jest działaniem wewnętrznym, to z zasady szufladkowej istnieją takie $a_k \neq a_l$, że $aa_k = aa_l$. To stwierdzenie jest jednak sprzecznością na mocy lematu \ref{l:cancellation_property}, ponieważ rozważamy pierścienie całkowite, w których nie ma dzielników zera.
\end{proof}

\begin{example}
    Przykłady pierścieni i ciał:
    \begin{itemize}
        \item $(\ZZ, +, \cdot)$ -- pierścień całkowity, który nie jest ciałem (nie ma dzielników zera, ale często elementy odwrotne nie zawierają się w zbiorze $\ZZ$),
        \item $(\QQ, +, \cdot)$ -- ciało przemienne liczb wymiernych,
        \item $(\RR, +, \cdot)$ -- ciało przemienne liczb rzeczywistych,
        \item $(\CC, +, \cdot)$ -- ciało przemienne liczb zespolonych,
        \item $(\ZZ_n, +_n, \cdot_n)$ -- pierścień przemienny z jedynką.
    \end{itemize}
\end{example}

\begin{corollary}[z twierdzenia \ref{t:prime_n->group}]
    Pierścień $(\ZZ_n, +_n, \cdot_n)$ jest ciałem wtedy i tylko wtedy, gdy $n$ jest liczbą pierwszą.
\end{corollary}

\subsection{Morfizmy}
\begin{definition}
    \label{d:homomorphism}
    Homomorfizmem grupy $(A_1, +)$ w grupę $(A_2, \oplus)$ jest takie odwzorowanie $h : A_1 \to A_2$, że
    $$ \forall x, y \in A_1 : h(x + y) = h(x) \oplus h(y). $$
\end{definition}

\begin{fact}
    Jeśli $h : A_1 \to A_2$ jest homomorfizmem grupy $(A_1, +)$ w $(A_2, \oplus)$, to
    \begin{enumerate}[noitemsep,nolistsep]
        \item $e \in A_1$ jest elementem neutralnym w $(A_1, +)$ $\Longrightarrow$ $h(e) \in A_2$ jest elementem neutralnym w $(A_2, \oplus)$,
        \item $\forall x \in A_1 : h(x') = h(x)'$.
    \end{enumerate}
\end{fact}

\begin{definition}
    Izomorfizm między grupami $(A_1, +), (A_2, \oplus)$ jest homomorfizmem bijektywnym. Jeśli taki izomorfizm istnieje, to dwie grupy nazywamy izomorficznymi.
\end{definition}

\begin{definition}
    Automorfizm to izomorfizm struktury na samą siebie.
\end{definition}

Analogicznie definiujemy morfizmy między pierścieniami i ciałami (wtedy równość z definicji \ref{d:homomorphism} musi zachodzić dla obydwu działań).

\begin{example}
    Przykłady morfizmów:
    \begin{itemize}
        \item $h(x) = x^2$ jest homomorfizmem grupy $(\RR \setminus \{0\}, \cdot)$ w $(\RR_+, \cdot)$,
        \item $h(x) = e^x$ jest izomorfizmem grupy $(\RR, +)$ w $(\RR_+, \cdot)$, ponieważ
        $$ h(x + y) = e^{x + y} = e^x \cdot e^y = h(x) \cdot g(y), $$
        \item $h(z) = \ol{z}$ jest automorfizmem grupy $(\CC, +).$
    \end{itemize}
\end{example}

Na podobnej zasadzie jak w przykładzie drugim, można pokazać izomorfizm grupy $(\ZZ_n, +_n)$ z grupą pierwiastków $n$-tego stopnia z jedności względem mnożenia $(\mu_n(\CC), \cdot)$. Biorąc funkcję $h(x) = \cos(\frac{2\pi}{n}x) + i\sin(\frac{2\pi}{n}x)$, mamy
\begin{align*} h(x + y) &= \cos(\tfrac{2\pi}{n}(x + y)) + i\sin(\tfrac{2\pi}{n}(x + y)) \\
    &= \left(\cos(\tfrac{2\pi}{n}x) + i\sin(\tfrac{2\pi}{n}x)\right) \cdot \left(\cos(\tfrac{2\pi}{n}y) + i\sin(\tfrac{2\pi}{n}y)\right) = h(x) \cdot h(y)\end{align*}
