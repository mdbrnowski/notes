\begin{theorem}
    \label{t:rank}
    Maksymalna liczba liniowo niezależnych kolumn (wektorów z $\KK^m$) dowolnej macierzy $A \in M_{m\times n}(\KK)$ jest równa maksymalnej liczbie liniowo niezależnych wierszy (wektorów z $\KK^n$).
\end{theorem}
\begin{proof}
    TODO
\end{proof}

\begin{definition}
    \label{d:rank}
    Rząd macierzy to maksymalna liczba liniowo niezależnych kolumn (lub wierszy). Rząd macierzy $A$ oznaczamy przez $\rank (A)$.
\end{definition}

Z twierdzenia \ref{t:rank} i definicji \ref{d:rank} wynika, że
\[ \rank(A) = \rank(A^T). \]

\begin{definition}
    Macierz schodkowa to macierz, której pierwsze niezerowe elementy (schodki) kolejnych niezerowych wierszy znajdują się w coraz dalszych kolumnach, a wiersze zerowe umieszczone są najniżej.
\end{definition}

\begin{fact}
    Rząd macierzy schodkowej jest równy liczbie jej schodków.
\end{fact}

\begin{definition}
    Operacje elementarne na macierzach to:
    \begin{itemize}
        \item zamiana miejscami wierszy (kolumn) macierzy,
        \item dodanie do wiersza (kolumny) kombinacji liniowej pozostałych wierszy (kolumn),
        \item pomnożenie wiersza przez niezerowy skalar.
    \end{itemize}
    Jeśli macierz $B$ można otrzymać z macierzy $A$ za pomocą operacji elementarnych, to będziemy oznaczać $A \sim B$.
\end{definition}

\begin{fact}
    Rząd macierzy nie zmienia się pod wpływem operacji elementarnych.
\end{fact}
\begin{proof}
    Wynika z twierdzenia \ref{t:linear independence}.
\end{proof}

Każdą macierz można łatwo doprowadzić do postaci schodkowej, za pomocą metody \vocab{eliminacji Gaussa}, która polega na stosowaniu operacji elementarnych na wierszach, ,,pozbywając się'' niezerowych elementów z dolnego trójkąta. W ten sposób można odczytać jej rząd oraz, za pomocą wniosku \ref{c:determinant in triangular matrix}, obliczyć jej wyznacznik (jeśli jest kwadratowa)\footnote{warto jednak zwrócić uwagę, że przy obliczaniu wyznacznika lepiej nie mnożyć wierszy i nie zamieniać ich miejscami, bo te operacje wpływają na wyznacznik}. 

\begin{example}
    Obliczyć wyznacznik macierzy
    \[ \begin{bmatrix}
        1 & 3 & 1 \\
        1 & 1 & -1 \\
        3 & 11 & 6
    \end{bmatrix}. \]
\end{example}
\begin{solution}
    \[ \begin{vmatrix}
        1 & 3 & 1 \\
        1 & 1 & -1 \\
        3 & 11 & 6
    \end{vmatrix} = \begin{vNiceMatrix}[create-medium-nodes]
        \CodeBefore[create-cell-nodes]
            \begin{tikzpicture}[name suffix = -medium]
                \node [highlight = (2-1) (2-3)] {} ;
                \node [highlight = (3-1) (3-3)] {} ;
            \end{tikzpicture}
        \Body
        1 & 3 & 1 \\
        0 & -2 & -2 \\
        0 & 2 & 3
    \end{vNiceMatrix} = \begin{vNiceMatrix}[create-medium-nodes]
        \CodeBefore[create-cell-nodes]
            \tikz[name suffix = -medium] \node [highlight = (3-1) (3-3)] {} ;
        \Body
        1 & 3 & 1 \\
        0 & -2 & -2 \\
        0 & 0 & 1
    \end{vNiceMatrix} = -2 \]
\end{solution}

\begin{theorem}
    \label{t:rank = max order of nonzero minor}
    Rząd macierzy $A$ jest równy największemu ze stopni niezerowych minorów tej macierzy.
\end{theorem}
\begin{proof}
    TODO
\end{proof}

\begin{example}
    Obliczyć rząd macierzy
    \[ A = \begin{bmatrix}
        1 & 3 & 4 \\
        3 & 4 & 1 \\
        1 & 2 & 7 \\
        3 & 5 & -1
    \end{bmatrix}. \]
\end{example}
\begin{solution}
    Obliczmy minor
    \[ \begin{vmatrix}
        1 & 3 & 4 \\
        3 & 4 & 1 \\
        3 & 5 & -1
    \end{vmatrix} = \begin{vmatrix}
        1 & 3 & 4 \\
        3 & 4 & 1 \\
        0 & 1 & -2
    \end{vmatrix} = (-8) + 0 + 12 - 0 - 1 - (-18) = 21 \neq 0. \]
    Ten minor jest niezerowy i jednocześnie ma największy stopień (bo wykreśliliśmy tylko jeden wiersz), więc na mocy twierdzenia \ref{t:rank = max order of nonzero minor} $\rank A = 3$.

    Dla pewności można pokazać również inną metodę --- eliminację Gaussa:
    \[ \begin{bmatrix}
        1 & 3 & 4 \\
        3 & 4 & 1 \\
        1 & 2 & 7 \\
        3 & 5 & -1
    \end{bmatrix} \sim \begin{bmatrix}
        1 & 3 & 4 \\
        0 & -5 & -11 \\
        0 & -1 & 3 \\
        0 & -4 & -13
    \end{bmatrix} \sim \begin{bmatrix}
        1 & 3 & 4 \\
        0 & 0 & -26 \\
        0 & -1 & 3 \\
        0 & 0 & -25
    \end{bmatrix} \sim \begin{bmatrix}
        1 & 3 & 4 \\
        0 & -1 & 3 \\
        0 & 0 & 1 \\
        0 & 0 & 0 \\
    \end{bmatrix} \]
    \[ \therefore \rank A = \rank \left[\begin{smallmatrix}
        1 & 3 & 4 \\
        0 & -1 & 3 \\
        0 & 0 & 1 \\
        0 & 0 & 0 \\
    \end{smallmatrix}\right] = 3. \]
\end{solution}