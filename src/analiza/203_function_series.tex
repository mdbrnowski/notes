Podobnie do szeregów liczbowych, szeregi funkcyjne to para $((f_n(x))_{n\in\NN}, (S_n(x))_{n\in\NN})$: ciąg funkcyjny oraz ciąg sum częściowych ciągu funkcyjnego. Taki szereg jest zbieżny (punktowo / jednostajnie) do sumy szeregu $S$, jeśli ciąg $(S_n(x))$ jest zbieżny (częściowo / jednostajnie) do $S$.

Analogicznie do twierdzenia \ref{t:uniform convergence implies pointwise convergence}, warukiem koniecznym zbieżności jednostajnej szeregu jest jego zbieżność punktowa.

Z kolei w analogii do twierdzenia \ref{t:necessary condition of convergence}, warunkiem koniecznym zbieżności (punktowej / jednostajnej) szeregu $\sum_{n=1}^\infty f_n(x)$ jest zbieżność (punktowa / jednostajna) ciągu funkcyjnego $(f_n(x))$ do zera, to znaczy
\[ \sum_{n=1}^\infty f_n(x) \rightarrow S \Longrightarrow f_n(x) \rightarrow 0 \equiv f \]
oraz
\[ \sum_{n=1}^\infty f_n(x) \rightrightarrows S \Longrightarrow f_n(x) \rightrightarrows 0 \equiv f. \]

\begin{theorem}[kryterium Weierstrassa]
    Jeśli istnieje taki ciąg $(a_n)$, że dla każdego $n \in \NN$ i dla każdego $x \in X \subset \RR$ mamy nierówność
    \[ |f_n(x)| \leq a_n \]
    oraz szereg $\sum_{n=1}^\infty a_n$ jest zbieżny, to szereg funkcyjny
    \[ \sum_{n=1}^\infty f_n(x) \]
    jest jednostajnie zbieżny na $X$.
\end{theorem}

Zachodzi twierdzenie o ciągłości, analogiczne do twierdzenia \ref{t:continuous limit}.

\begin{theorem}
    Jeśli szereg $\sum_{n=1}^\infty f_n(x)$ jest ciągiem funkcji ciągłych i jest jednostajnie zbieżny $\sum_{n=1}^\infty f_n(x) \rightrightarrows S(x)$, to funkcja $S$ jest ciągła.
\end{theorem}

Zachodzą również twierdzenia o różniczkowalności i całkowalności, analogiczne do twierdzeń \ref{t:differentiable limit} i \ref{t:integrable limit}.

\begin{theorem}
    Niech $(f_n(x))$ będzie ciągiem funkcji różniczkowalnych. Jeśli szereg $\sum_{n=1}^\infty f_n(x)$ jest zbieżny na $X$, a szereg $\sum_{n=1}^\infty f_n'(x)$ jest jednostajnie zbieżny na $X$, to dla każdego $x \in X$ zachodzi
    \[ \left(\sum_{n=1}^\infty f_n(x)\right)' = \sum_{n=1}^\infty f_n'(x). \]
\end{theorem}

\begin{theorem}
    Niech $(f_n(x))$ będzie ciągiem funkcji całkowalnych. Jeśli szereg $\sum_{n=1}^\infty f_n(x)$ jest jednostajnie zbieżny na $X$, to dla każdych $x_1, x_2 \in X$ zachodzi
    \[ \int_{x_1}^{x_2}\left(\sum_{n=1}^\infty f_n(x)\right)\d x = \sum_{n=1}^\infty \left(\int_{x_1}^{x_2}f_n(x)\d x\right). \]
\end{theorem}

\subsection{Szeregi potęgowe}
\begin{definition}
    \label{d:power series}
    Szereg potęgowy o środku w punkcie $c$ to szereg funkcyjny
    \[ \sum_{n=1}^\infty a_n(x - c)^n, \]
    gdzie $a_n, x, c \in \CC$.
\end{definition}

\begin{theorem}
    Jeśli szereg potęgowy
    \[ \sum_{n=1}^\infty a_n(x - c)^n \]
    jest zbieżny dla pewnego $x_1$, to jest zbieżny dla wszystkich $x_2$ takich, że
    \[ |x_2 - c| < |x_1 - c|, \]
    a jeśli nie jest zbieżny dla pewnego $x_1$, to nie jest zbieżny dla wszystkich $x_2$ takich, że
    \[ |x_2 - c| > |x_1 - c|. \]
\end{theorem}

Powyższe twierdzenie każe nam podzielić płaszczyznę zespoloną (względem danego szeregu potęgowego) na trzy rozłączne zbiory. Formalnie, jeśli weźmiemy
\[ r = \sup\left\{|x - c| : \text{ szereg } \sum_{n=1}^\infty a_n(x - c)^n \text{ jest zbieżny}\right\}, \]
to zbiór
\[ \{x \in \CC : |x - x_0| < r\} \]
nazwiemy \vocab{kołem zbieżności}. Dla wszystkich elementów z tego zbioru dany szereg jest zbieżny. Dla elementów na brzegu tego koła zbieżność jest nieokreślona, a dla elementów poza nim dany szereg nie jest zbieżny. Liczba $r$ to \vocab{promień zbieżności}. Dla $x=c$ dany szereg jest zbieżny.

\begin{remark*}
    Jeśli przyjmiemy w definicji szeregu potęgowego (\ref{d:power series}), że $a_n, x, c \in \RR$, to koło zbieżności staje się \vocab{przedziałem zbieżności}, a nieokreśloną zbieżność mamy tylko dla dwóch elementów: $c - r$ oraz $c + r$.
\end{remark*}

\begin{theorem}[Cauchy'ego-Hadamarda]
    \label{t:Cauchy-Hadamard}
    Promień zbieżności jest dany jako
    \[ r = \frac{1}{\limsup\limits_{n\lthen\infty}\sqrt[n]{|a_n|}}, \]
    gdzie $r = \frac{1}{0}$ interpretujemy jako $r = \infty$, a $r = \frac{1}{\infty}$ jako $r = 0$.
\end{theorem}

Można podać dwa słabsze twierdzenia, które jednak często łatwiej jest stosować:
\[ r = \frac{1}{\lim\limits_{n\lthen\infty} \left|\frac{a_{n+1}}{a_n}\right|} \hspace{1em}\Longrightarrow\hspace{1em} r = \frac{1}{\lim\limits_{n\lthen\infty}\sqrt[n]{|a_n|}} \hspace{1em}\Longrightarrow\hspace{1em} (\ref{t:Cauchy-Hadamard}). \]