\begin{definition}
    Pochodną funkcji $f$ nazwiemy taką funkcję $f'$, że
    \[ f'(x) = \lim_{h\lthen 0}\frac{f(x + h) - f(x)}{h}. \]
    Jeśli wartość $f'(x_0)$ istnieje, to mówimy, że funkcja $f$ jest różniczkowalna w punkcie $x_0$.
\end{definition}

Oprócz notacji Lagrange'a ($f'$) stosuje się również notację Leibniza ($f' = \frac{df(x)}{dx})$.

\begin{theorem}
    Jeśli $f$ jest różniczkowalna w punkcie $x_0$, to jest w tym punkcie ciągła.
\end{theorem}
\begin{proof}
    \[ \lim_{h \lthen 0}(f(x_0 + h) - f(x_0)) = \lim_{h \lthen 0}\frac{f(x_0 + h) - f(x_0)}{h} \cdot h = f'(x_0) \cdot h = 0 \]
    więc
    \[\ \lim_{h\lthen 0}f(x_0 + h) = f(x_0), \]
    ergo $f$ jest ciąła w $x_0$.
\end{proof}

Pochodną funkcji w punkcie możemy interpretować jako nachylenie stycznej do wykresu funkcji w tym punkcie. Równanie takiej stycznej ma postać
\begin{equation}
    y - f(x_0) = f'(x_0)(x - x_0)
\end{equation}

TODO rysunek

\begin{theorem}[wzory pochodnych podstawowych funkcji]
    Zachodzą równości:
    \begin{enumerate}
        \item $\ddx c = 0$
        \item $\ddx x^r = rx^{r-1}$
        \item $\ddx \sin x = \cos x$
        \item $\ddx \cos x = -\sin x$
        \item $\ddx \tan x = \frac{1}{\cos^2 x} = 1 + \tan^2 x$
        \item $\ddx \cot x = - \frac{1}{\sin^2 x} = -1 - \cot^2 x$
        \item $\ddx e^x = e^x$
        \item $\ddx a^x = a^x \ln a$
    \end{enumerate}
\end{theorem}
\begin{proof}
    TODO
\end{proof}

\begin{theorem}[o pochodnej funkcji złożonej]
    \[ f(g(x))' = f'(g(x)) \cdot g'(x) \]
\end{theorem}
\begin{proof}
    \[ \frac{df(g(x))}{dx} = \frac{df(g(x))}{dg(x)}\cdot\frac{dg(x)}{dx} = f'(g(x)) \cdot g'(x) \]
\end{proof}

\begin{theorem}[o pochodnej funkcji odwrotnej]
    Dana jest bijekcja $f: U \lthen V$, gdzie $U$ jest otoczeniem punktu $x_0$, a $V$ -- otoczeniem $y_0 = f(x_0)$. Jeśli $f$ jest różniczkowalna w $x_0$ oraz $f'(x_0) \neq 0$, to
    \[ \left(f^{-1}\right)'(y_0) = \frac{1}{f'(x_0)}. \]
\end{theorem}
\begin{proof}
    TODO
\end{proof}

\begin{example}
    Obliczyć pochodną funkcji $\arctan$.
\end{example}
\begin{solution}
    Funkcja $\tan : (-\frac{\pi}{2}, \frac{\pi}{2}) \lthen (-\infty, \infty)$ jest bijekcją oraz jest różniczkowalna na całym przedziale. Ponadto, jej pochodna nigdy się nie zeruje. Mamy więc
    \[ \arctan'(\tan x) = \frac{1}{\tan'(x)} = \frac{1}{\frac{1}{\cos^2 x}} = \cos^2 x \]
    \[ \arctan'(x) = \cos^2(\arctan x) \]
    \[ \arctan'(x) = \frac{\cos^2(\arctan x)}{\sin^2(\arctan x)} \cdot \sin^2(\arctan x) \]
    \[ \arctan'(x) = \frac{1}{\tan^2(\arctan x)} \cdot (1 - \arctan'(x)) \]
    \[ \arctan'(x) = \frac{1}{x^2} - \arctan'(x)\frac{1}{x^2} \]
    \[ \arctan'(x) = \frac{\frac{1}{x^2}}{1 + \frac{1}{x^2}} = \frac{1}{x^2}\cdot\frac{x^2}{x^2 + 1} = \frac{1}{x^2 + 1} \]
\end{solution}

\begin{theorem}
    \[ \ddx \ln x = \frac{1}{x} \]
\end{theorem}
\begin{proof}
    \begin{gather*}
        \ddx x = 1 \\
        \ddx e^{\ln x} = e^{\ln x} \ddx \ln x = 1 \\
        x \ddx \ln x = 1 \\
        \ddx \ln x = \frac{1}{x}
    \end{gather*}
\end{proof}

\begin{theorem}[o zerowaniu się pochodnej w punkcie, w którym funkcja przyjmuje ekstremum]
    \label{t:zero derivative in extremum}
    Jeśli funkcja $f$ jest ciągła na przedziale $[a, b]$, różniczkowalna na przedziale $(a, b)$ oraz istnieje takie $x_0 \in (a, b)$, że
    \[ f(x_0) = \max_{x \in [a, b]} f(x) \text{ lub } f(x_0) = \min_{x \in [a, b]} f(x), \]
    to
    \[ f'(x_0) = 0. \]
\end{theorem}
\begin{proof}
    Przypadek z minimum jest analogiczny, wykażemy tylko dla maksimum. Dla każdego $x < x_0$ mamy
    \[ f(x) \leq f(x_0) \]
    \[ \frac{f(x) - f(x_0)}{x - x_0} \geq 0 \]
    \[ \lim_{x\lthen x_0^-}\frac{f(x) - f(x_0)}{x - x_0} \geq 0 \]
    \[ f'_-(x_0) \geq 0, \]
    a dla każdego $x > x_0$ analogicznie
    \[ f'_+(x_0) \leq 0, \]
    więc $f(x_0) = 0$.
\end{proof}

\begin{theorem}[Rolle'a]
    \label{t:Rolle}
    Jeśli funkcja $f$ jest ciągła na przedziale $[a, b]$ oraz różniczkowalna na przedziale $(a, b)$, to z
    \[ f(a) = f(b) \]
    wynika, że istnieje takie $c \in (a, b)$, że
    \[ f'(c) = 0. \]
\end{theorem}
\begin{proof}
    Wynika z twierdzenia o zerowaniu się pochodnej (\ref{t:zero derivative in extremum}) oraz twierdzania Weierstrassa (\ref{t:Weierstrass}).
\end{proof}

\begin{theorem}[Lagrange'a]
    Jeśli funkcja $f$ jest ciągła na przedziale $[a, b]$ oraz różniczkowalna na przedziale $(a, b)$, to istnieje takie $c \in (a, b)$, że
    \[ f'(c) = \frac{f(b) - f(a)}{b - a}. \]
\end{theorem}
\begin{proof}
    Niech $h(x) = f(x) - \frac{f(b) - f(a)}{b - a}(x - a)$. Z twierdzenia Rolle'a (\ref{t:Rolle}) dla funkcji $h$ wynika teza.
\end{proof}

\begin{theorem}[Cauchy'ego]
    Jeśli funkcje $f, g$ są ciągłe na przedziale $[a, b]$ oraz różniczkowalne na przedziale $(a, b)$, to istnieje takie $c \in (a, b)$, że
    \[ g'(c) (f(b) - f(a)) = f'(c) (g(b) - g(a)). \]
\end{theorem}
\begin{proof}
    Niech $h(x) = g(x)(f(b) - f(a)) - f(x)(g(b) - g(a))$. Z twierdzenia Rolle'a (\ref{t:Rolle}) dla funkcji $h$ wynika teza.
\end{proof}

Twierdzenia Rolle'a, Lagrange'a oraz Cauchy'ego nazywamy \vocab{twierdzeniami o wartości średniej}.

\begin{theorem}[reguła de l'Hospitala]
    \label{t:Hospital}
    Jeśli funkcje $f, g$ są różniczkowalne w pewnym sąsiedztwie $S$ punktu $x_0 \in \ol{\RR}$ oraz
    \begin{enumerate}
        \item dla każdego $x \in S$ zachodzi $g(x) \neq 0$ oraz $g'(x) \neq 0$,
        \item $\lim\limits_{x\lthen x_0} f(x) = \lim\limits_{x\lthen x_0} g(x) \in \{0, \infty, -\infty\}$,
        \item \label{enum:Hospital exists} istnieje $\lim\limits_{x\lthen x_0}\frac{f'(x)}{g'(x)}$,
    \end{enumerate}
    to prawdą jest, że
    \[ \lim_{x\lthen x_0}\frac{f(x)}{g(x)} = \lim_{x\lthen x_0}\frac{f'(x)}{g'(x)}. \]
\end{theorem}
\begin{proof}
    TODO
\end{proof}
\begin{remark}
    Warunek \ref{enum:Hospital exists} jest bardzo ważny; gdybyśmy regułę de l'Hospitala wykorzystali do obliczenia granicy
    \[ \lim_{x\lthen\infty}\frac{x}{x + \sin x} \]
    wyszłoby nam, że
    \[ \lim_{x\lthen\infty}\frac{1}{1 + \cos x} \]
    nie istnieje, bo ma podciągi zbieżne do $1$ i $\frac{1}{2}$. Moglibyśmy (błędnie stosując wspomianą regułę) wyciągnąć wniosek, że dana wcześniej granica również nie istnieje, co jednak jest nieprawdą, bo jest równa $1$ na mocy twierdzenia o trzech funkcjach (\ref{t:squeeze theorem}).
\end{remark}

\subsection{Wzory Taylora i Maclaurina}
\begin{theorem}[Taylora]
    \label{t:Taylor}
    Jeśli funkcja $f$ jest $n$-krotnie różniczkowalna na przedziale $[a, b]$, to dla $x \in (a, b)$ zachodzi
    \[ f(x) = f(a) + \frac{f'(a)}{1!}(x - a) + \frac{f''(a)}{2!}(x - a)^2 + \ldots + \frac{f^{(n-1)}(a)}{(n-1)!}(x - a)^{n-1} + R_n(x, a), \]
    gdzie
    \[ R_n(x, a) = \frac{f^{(n)}(c)}{n!}(x - a)^n \]
    dla $c \in [a, x]$ jest nazywane \vocab{resztą Lagrange'a}.
\end{theorem}

Resztę Lagrange'a możemy również zapisać w postaci
\[ R_n(x, a) = \frac{f^{(n)}(a + \theta(x - a))}{n!}(x - a)^n \]
dla $\theta \in [0, 1]$.

\begin{definition}
    We wzorze Taylora (\ref{t:Taylor}), jeśli $a = 0$, to ten wzór nazywamy \vocab{wzorem Maclaurina}.
\end{definition}

\begin{example}
    Oblicz $\cos\frac{1}{40}$ z dokładnością do $10^{-9}$.
\end{example}
\begin{solution}
    Skorzystamy ze wzoru Maclaurina przy $f = \cos$.
    \[ \cos\frac{1}{40} = \cos(0) + \frac{-\sin(0)}{1!}\left(\frac{1}{40}\right) + \ldots + R_n, \]
    gdzie
    \begin{align*}
        \left|R_n\right| &\leq 10^{-9} \\
        \left|\frac{\cos^{(n)}(c)}{n!}\left(\frac{1}{40}\right)^n\right| &\leq 10^{-9}.
    \end{align*}
    Dla $n = 5$ możemy oszacować
    \[ \left|\frac{-\sin(c)}{5!}\left(\frac{1}{40}\right)^5\right| = \frac{\sin(c)}{5! \cdot 40^5} \leq \frac{c}{120 \cdot 40^5}. \]
    A skoro $c \in [0, \frac{1}{40}]$, to
    \[ \frac{c}{120 \cdot 40^5} \leq \frac{1}{120 \cdot 40^6} = \frac{1}{12 \cdot 16^3} \cdot 10^{-7} < 10^{-9}, \]
    więc
    \begin{align*}
        \cos\frac{1}{40} &\approx 1 + 0 + \frac{-1}{2!}\left(\frac{1}{40}\right)^2 + 0 + \frac{1}{4!}\left(\frac{1}{40}\right)^4 \\
        &= 1 - \frac{1}{2 \cdot 40^2} + \frac{1}{24 \cdot 40^4}.
    \end{align*}
    Pomagając sobie kalkulatorem można sprawdzić, że nasz błąd wynosi około $3.39\cdot 10^{-13}$.
\end{solution}