\documentclass[11pt]{scrartcl}
\usepackage[pretty,polish]{mystd}
\title{Algebra}
\author{Michał Dobranowski}
\date{semestr zimowy 2022 \\ v0.2}

\begin{document}
    \maketitle
    \begin{abstract}
        Poniższy skrypt zawiera materiał obejmujący wykłady z Algebry prowadzone przez dr hab. Jakuba Przybyło na I semestrze Informatyki na AGH oraz tematy, które uznałem za warte uwagi podczas własnych studiów nad tematem.
    \end{abstract}
    \tableofcontents
    \eject

    \section{Liczy zespolone}
    \begin{definition}
        Liczba zespolona $z$ to uporządkowana para liczb rzeczywistych. Pierwszy element tej pary to \vocab{część rzeczywista}, ozaczana symbolem $\Re(z)$, a drugi to \vocab{część urojona}, oznaczana symbolem $\Im(z)$. Zbiór liczb zespolonych oznaczamy przez $\CC$.
    \end{definition}

    Liczby zespolone można reprezentować w kilku postaciach, jedna z nich to \vocab{postać algebraiczna}. Używając jej, liczba $z = (x, y)$ jest zapisywana jako
    $$ z = x + iy, $$
    gdzie $i$ nazywamy \vocab{jednostką urojoną}, która spełnia
    $$ i^2 = -1. $$

    \subsection{Działania na liczbach zespolonych}
    Niech $z_1 = x_1 + iy_1$ oraz $z_2 = x_2 + iy_2$. Określamy:
    \begin{itemize}
        \item dodawanie $z_1 + z_2 = x_1 + x_2 + i(y_1 + y_2)$
        \item mnożenie $\begin{aligned}[t] z_1z_2 &= x_1x_2 + ix_1y_2 + ix_2y_1 + i^2y_1y_2 \\ &= x_1x_2 - y_1y_2 + i(x_1y_2 + x_2y_1)\end{aligned}$
    \end{itemize}

    \begin{corollary}
        Dodawanie i mnożenie liczb zespolonych jest przemienne i łączne. Mnożenie jest rozdzielne względem dodawania.
    \end{corollary}

    \begin{definition}
        Sprzężenie liczby zespolonej $z = x + iy$ to liczba $\ol{z} = x - iy$.
    \end{definition}

    \begin{definition}
        \label{d:magnitude}
        Moduł liczby zespolonej $z = x + iy$ to liczba $|z| = \sqrt{x^2 + y^2}$.
    \end{definition}

    Zachodzi pewna własność, wynikająca ze wzoru skróconego mnożenia:
    $$ z\ol{z} = (x + iy)(x - iy) = x^2 - i^2y^2 = x^2 + y^2 $$
    \begin{equation}
        z\ol{z} = |z|^2
    \end{equation}

    Powyższa liczba jest liczbą rzeczywistą, więc znaleźliśmy prosty sposób na dzielenie liczb zespolonych przez siebie, mnożąc licznik i mnianownik przez sprzężenie mianownika. Na przykład:
    $$ \frac{1 + 2i}{-1 - i} = \frac{(1 + 2i)(-1 + i)}{(-1 - i)(-1 + i)} = \frac{-3 -i}{2} = \frac{-3}{2} - \frac{i}{2}. $$

    \begin{lemma}
        Oprócz $z\ol{z} = |z|^2$, zachodzą również równości:
        \begin{itemize}
            \item $|\ol{z}| = |z|$
            \item $\ol{z_1 + z_2} = \ol{z_1} + \ol{z_2}$
            \item $\ol{z_1z_2} = \ol{z_1}\cdot\ol{z_2}$
            \item $|z_1z_2| = |z_1||z_2|$
        \end{itemize}
    \end{lemma}
    Ich dowody można w łatwy sposób przeprowadzić z definicji poszczególnych działań.

    \subsection{Interpretacja geometryczna liczb zespolonych}
    Liczby zespolone można interpretować jako punkty na \vocab{płaszczyźnie zespolonej}. Dla przykładu liczba $z = 3 + 2i$.

    \begin{center}
        \begin{tikzpicture}
            \tkzInit[xmin=-.7, xmax=3.7, ymin=-.7, ymax=2.7]
            \tkzDefPoints{0/0/O,3/2/z}
            \tkzGrid
            \tkzDrawX[label=$\Re$,thick] \tkzDrawY[label=$\Im$,thick]
            \tkzDrawSegments(O,z)
            \tkzDrawPoints(z)
            \tkzLabelPoints[above right](z)
        \end{tikzpicture}
    \end{center}

    \begin{fact}
        Moduł liczby zespolonej $z$ to długość wektora wodzącego tej liczby na płaszczyźnie zespolonej.
    \end{fact}
    \begin{proof}
        Wynika to z twierdzenia Pitagorasa oraz definicji modułu (\ref{d:magnitude}).
    \end{proof}

    Możemy wyprowadzić \vocab{postać trygonometryczną} liczby zespolonej, która, zamiast dwóch współrzędnych, będzie operować na długości wektora wodzącego oraz kącie skierowanym. Mamy więc
    $$ z = |z|(\cos\varphi + i\sin\varphi) $$
    gdzie $\varphi$ to miara kąta skierowanego między wektorem wodzącym liczby zespolonej $z$ a osią liczb rzeczywistych. Ten kąt nazywany jest \vocab{argumentem} i oznaczany przez $\Arg(z)$. Argument nie jest określony jednoznacznie -- dowolne dwa argumenty jednej liczby różnią się o wielokrotność $2\pi$. Jeśli argument jest w przedziale $[0, 2\pi)$, to mówimy, że jest to \vocab{argument główny} liczby $z$ i oznaczamy $\arg(z)$.

    Za pomocą podstawowej trygonometrii możemy łatwo zamieniać postać algebraiczną i trygonometryczną między sobą.

    \begin{center}
        \begin{tikzpicture}
            \tkzInit[xmin=-.7, xmax=3.7, ymin=-.7, ymax=2.7]
            \tkzDefPoints{0/0/O,1/0/A,3/2/z}
            \tkzDefPointBy[projection=onto O--A](z) \tkzGetPoint{z'}
            \tkzGrid
            \tkzDrawX[label=$\Re$,thick] \tkzDrawY[label=$\Im$,thick]
            \tkzDrawSegment[dim={$|z|$,2mm,}, dim style/.style={sloped,dashed}](O,z)
            \tkzDrawSegment[dim={$|z|\cos\varphi$,-2mm,}, dim style/.style={sloped,dashed}](O,z')
            \tkzDrawSegment[dim={$|z|\sin\varphi$,2mm,}, dim style/.style={sloped,dashed}](z,z')
            \tkzMarkAngle[size=1.1](z',O,z)
            \tkzLabelAngle[pos=.8](z',O,z){$\varphi$}
            \tkzDrawPoints(z)
            \tkzLabelPoints[above right](z)
        \end{tikzpicture}
    \end{center}

    \begin{equation}
        \Re{z} = |z|\cos\varphi, \hspace{2em} \Im{z} = |z|\sin\varphi
    \end{equation}

    Na potrzeby dalszych rozważań przyjmujemy, że $\arg(0) = 0$.

    \begin{fact}
        Odległość między liczbami $z_1$ i $z_2$ na płaszczyźnie zespolonej wynosi $|z_1 - z_2|$.
    \end{fact}

    \begin{lemma}
        Zachodzą następujące nierówności:
        \begin{itemize}
            \item $|z_1 + z_2| \leq |z_1| + |z_2|$
            \item $||z_1| - |z_2|| \leq |z_1 - z_2|$
        \end{itemize}
    \end{lemma}

    Możemy łatwo mnożyć dwie liczby zespolone w postaci trygonometrycznej przez siebie za pomocą poniższego wzoru.
    \begin{equation}
        \label{eq:complex_prod}
        \begin{aligned}
            z_1 \cdot z_2 &= |z_1|(\cos\varphi_1 + i\sin\varphi_1)|z_2|(\cos\varphi_2 + i\sin\varphi_2) \\
                          &= |z_1||z_2|(\cos\varphi_1\cos\varphi_2 - \sin\varphi_1\sin\varphi_2 + i(\cos\varphi_1\sin\varphi_2 + \sin\varphi_1\cos\varphi_2)) \\
                          &= |z_1||z_2|(\cos(\varphi_1 + \varphi_2) + i\sin(\varphi_1 + \varphi_2))
        \end{aligned}
    \end{equation}

    Stosując wzór \ref{eq:complex_prod} $n$ razy otrzymujemy dowód następującego twierdzenia.

    \begin{theorem}[Wzór de Moivre'a]
        \label{t:demoivre}
        Dla $z = |z|(\cos\varphi + i\sin\varphi)$ oraz $n \in \ZZ$ zachodzi równość
        $$ z^n = |z|^n(\cos n\varphi + i\sin n\varphi) $$
    \end{theorem}

    Wzór de Moivre'a zapewnia prosty sposób na potęgowanie liczb zespolonych. Dlatego, mając za zadanie obliczyć
    $$ (-2\sqrt{3} - 2i)^{16} $$
    najłatwiej będzie zmienić postać liczby do postaci trygonometrycznej, a następnie skorzystać z twierdzenia \ref{t:demoivre}.

    \subsection{Pierwiastkowanie liczb zespolonych}
    \begin{definition}[Pierwiastek liczby zespolonej]
        Jeśli $z$ jest liczbą zespoloną, to $\sqrt[n]{z}$ jest zbiorem wszystkich takich $w \in \CC$, że $w^n = z$.
    \end{definition}

    Korzystając ze wzoru de Moivre'a (twierdzenie \ref{t:demoivre}) łatwo wyprowadzić wzór
    \begin{equation}
        \label{eq:complex_root}
        \sqrt[n]{z} = \sqrt[n]{|z|}\left(\cos\frac{\varphi + 2k\pi}{n} + i\sin\frac{\varphi + 2k\pi}{n}\right), k \in \ZZ
    \end{equation}

    \begin{fact}
        Pierwiastków $n$-tego stopnia z $z \neq 0$ jest dokładnie $n$ i leżą one w równych odstępach na okręgu o środku w $0$ i promieniu $\sqrt[n]{|z|}$.
    \end{fact}
    \begin{proof}
        Dla $k \in \{0, 1, \ldots, n-1\}$ liczba z równości \ref{eq:complex_root} będzie przyjmować różne wartości (wynika to z okresowości funkcji trygonometrycznych). Liczby te będą na wspomnianym okręgu (to wynika wprost z postaci trygonometrycznej), a ich argumenty główne różnić będzie wielokrotność $\frac{2\pi}{n}$.
    \end{proof}

    \subsection{Postać wykładnicza}
    Postać $z = |z|e^{i\varphi}$ liczby zespolonej bedziemy nazywać \vocab{postacią wykładniczą} tej liczby.
    \begin{theorem}[Wzór Eulera]
        Dla każdego $\varphi \in \RR$ zachodzi
        $$ e^{i\varphi} = \cos\varphi + i\sin\varphi. $$
    \end{theorem}
    \begin{proof}
        Weźmy $z = \cos\varphi + i\sin\varphi$. Różniczkując po zmiennej $\varphi$ otrzymujemy
        $$ \frac{dz}{d\varphi} = -\sin\varphi + i\cos\varphi = iz $$
        $$ \therefore \frac{dz}{z} = id\varphi. $$
        Po obustronnym całkowaniu mamy
        $$ \int\frac{dz}{z} = \int id\varphi $$
        $$ \ln{z} = i\varphi + c $$
        $$ e^{\ln{z}} = e^{i\varphi + c} $$
        $$ z = e^{i\varphi + c}. $$
        Podstawiając $\varphi = 0$ otrzymujemy $1 = e^c$, skąd mamy $c = 0$, co kończy dowód.
    \end{proof}

    \section{Relacje}
    \begin{definition}
        Relacja to trójka $\sR = (X, \gr\sR, Y)$, gdzie $X$ i $Y$ są zbiorami, a $\gr \sR \subset X \times Y$.
    \end{definition}

    Zbiór $X$ nazywamy \vocab{naddziedziną}, $Y$ \vocab{zapasem}, $\gr \sR$ to \vocab{wykres} relacji. Piszemy, że $x \sR y$, jesli $(x, y) \in \gr\sR$. \vocab{Dziedzina} relacji $\sR$ to zbiór
    $$ D_\sR = \{x \in X : \exists y \in Y : x \sR y\}, $$
    a jej \vocab{przeciwdziedzina} to zbiór
    $$ \rotatebox[origin=c]{180}{$D$}_\sR = \{y \in Y : \exists x \in X : x \sR y\}. $$

    \begin{definition}
        Relacja odwrotna do relacji $\sR = (X, \gr\sR, Y)$ to taka relacja $\sR^{-1} = (Y, \gr\sR^{-1}, X)$, że
        $$ \gr \sR^{-1} = \{(y, x) \in Y \times X : (x, y) \in \gr\sR\}. $$
    \end{definition}

    \begin{definition}
        Złożeniem relacji $\sR = (X, \gr\sR, Y)$ z relacją $\sS = (Y, \gr\sS, Z)$ nazywamy relację
        $$ \sR \circ \sS = (X, \gr(\sR \circ \sS), Z), $$
        gdzie
        $$ \gr(\sR \circ \sS) = \{(x, z) \in X \times Z : \exists y \in Y : x \sR y \wedge y \sS z\}. $$
    \end{definition}

    \begin{definition}[rodzaje relacji]
        Relacja $\sR = (X, \gr\sR, X)$ jest:
        \begin{itemize}[--]
            \item \vocab{zwrotna} $\iff \forall x\in X : x \sR x$,
            \item \vocab{symetryczna} $\iff \forall x, y \in X : x \sR y \implies y \sR x$,
            \item \vocab{antysymetryczna} $\iff \forall x, y \in X : x \sR y \wedge y \sR x \implies x = y$,
            \item \vocab{asymetryczna} $\iff \forall x, y \in X : x \sR y \implies \neg y \sR x$,
            \item \vocab{przechodnia} $\iff \forall x, y, z \in X : x \sR y \wedge y \sR z \implies y \sR x$,
            \item \vocab{spójna} $\iff \forall x, y \in X : x \sR y \vee y \sR x \vee x = y$.
        \end{itemize}
    \end{definition}

    \begin{definition}
        Relacja równoważności to relacja $\sR = (X, \gr\sR, X)$, która jest zwrotna, przechodnia i symetryczna.
    \end{definition}

    \begin{definition}
        Jeżeli $(X, \sR)$ zbiorem z relacją równoważności, to dla każdego $x \in X$ klasą abstrakcji (klasą równoważności) tego elementu nazywamy zbiór
        $$ [x] = \{y \in X : x \sR y\}. $$
    \end{definition}

    \begin{definition}
        Zbiór ilorazowy relacji $\sR$ to zbiór klas abstrakcji tej relacji; przyjmujemy oznaczenie
        $$ X/\sR = \{[x] : x \in X\}. $$
    \end{definition}

    \begin{theorem}
        Niech $(X, \sR)$ będzie zbiorem z relacją równoważności. Wtedy
        $$ \forall x, y \in X : [x] \neq [y] \iff [x] \cap [y] = \emptyset . $$
    \end{theorem}
    \begin{proof}[Dowód wystarczalności]\renewcommand{\qedsymbol}{}
        Załóżmy przez sprzeczność, że $[x] \cap [y] \neq \emptyset$, a więc $\exists z \in X : x \sR z \wedge y \sR z$. Teraz weźmy dowolny element $a \in [x]$. Mamy więc $x \sR a$. Korzystając z symetryczności i przechodniości relacji $\sR$ mamy
        $$ a \sR x \wedge x \sR z \wedge z \sR y, $$
        $$ \therefore y \sR a. $$
        Z tego wynika, że $[x] \subset [y]$. Analogicznie (przyjmując na początku $a \in [y]$) dostaniemy, że $[y] \subset [x]$, wiec $[x] = [y]$, co jest sprzeczne z założeniem.
    \end{proof}
    \begin{proof}[Dowód konieczności]
        Załóżmy przez sprzeczność, że $[x] = [y]$. Wtedy $[x] \cap [y] = [x] \cap [x] = [x]$ nie może być zbiorem pustym, ponieważ ze zwrotności relacji $\sR$ wynika, że $x \sR x$, więc $[x]$ to zbiór przynajmniej jednoelementowy.
    \end{proof}

    Z powyższego twierdzenie wynika, że relacja równoważności w danym zbiorze $X$ dzieli ten zbiór na niepuste i rozłączne podzbiory, których suma daje cały zbiór $X$.

    \subsection{Porządki}
    \begin{definition}
        Porządek (częściowy) to relacja $\sR = (X, \gr\sR, X)$, która jest zwrotna, przechodnia i antysymetryczna. Zbiór $X$ nazywamy zbiorem (częściowo) uporządkowanym.
    \end{definition}

    \begin{definition}
        Porządek liniowy (totalny) to porządek, który jest spójny.
    \end{definition}

    Niech $(X, \preceq)$ będzie zbiorem z porządkiem częściowym. Wtedy \vocab{element największy} $\ol{M} \in X$ zbioru $X$ to taki element, że
    $$ \forall x \in X : x \preceq \ol{M}, $$
    a \vocab{element maksymalny} $M_{\max} \in X$ to taki element, że
    $$ \forall x \in X : (M_{\max} \preceq x) \implies (M_{\max} = x). $$

    \begin{remark}
        Analogicznie można zdefiniować \vocab{element najmniejszy} $\ol{m}$:
        $$ \forall x \in X : \ol{m} \preceq x $$
        oraz \vocab{element minimalny} $m_{\min}$:
        $$ \forall x \in X : x \preceq m_{\min} \implies (x = m_{\min}) $$
    \end{remark}

    \begin{theorem}
        \label{t:uniq_greatest}
        Niech $(X, \preceq)$ będzie zbiorem z porządkiem częściowym. Jeśli w zbiorze $X$ istnieje element największy, to jest on jedyny.
    \end{theorem}
    \begin{proof}
        Załóżmy przeciwnie, że istnieją dwa elementy największe $M_1, M_2$. Z definicji zachodzi
        $$ M_1 \preceq M_2 $$
        oraz
        $$ M_2 \preceq M_1, $$
        co jest sprzeczne z antysymetrycznością porządków.
    \end{proof}

    \begin{theorem}
        Niech $(X, \preceq)$ będzie zbiorem z porządkiem częściowym. Jeśli $M \in X$ jest elementem największym zbioru $X$, to jest on jedynym elementem maksymalnym tego zbioru.
    \end{theorem}
    \begin{proof}
        Skoro $M$ jest elementem największym, to poprzednik implikacji\footnote{to znaczy jej lewa strona.} w definicji elementu maksymalnego będzie prawdziwy tylko dla $x = M$, więc sama implikacja zawsze będzie prawdziwa.
    \end{proof}

    \begin{fact}
        \label{f:greatest=maximal}
        W zbiorach z porządkiem totalnym pojęcia elementu największego i maksymalnego oraz najmniejszego i minimalnego są tożsame ze sobą. Wynika to ze spójności porządków totalnych.
    \end{fact}

    Niech $(X, \preceq)$ będzie zbiorem uporządkowanym, a zbiór $A \subset X$ jego podzbiorem. Element $M \in X$ jest \vocab{majorantą} (ograniczeniem górnym) zbioru $A$ jeśli
    $$ \forall x \in A : x \preceq M. $$
    \vocab{Kresem górnym} (supremum) zbioru $A$ (w zbiorze $X$) jest element najmniejszy zbioru majorant. Oznaczamy go symbolem $$\sup A.$$

    \begin{remark}
        Analogicznie można zdefiniować \vocab{minorantę} (ograniczenie dolne) $m \in X$ zbioru $A \subset X$:
        $$ \forall x \in A : m \preceq x $$
        oraz \vocab{kres dolny} (infimum) tego zbioru (jest nim element największy zbioru minorant), który oznaczamy symbolem $$\inf A.$$
    \end{remark}

    \begin{theorem}
        \label{t:greatest=sup}
        Niech $(X, \preceq)$ będzie zbiorem z porządkiem częściowym oraz $A \subset X$. Jeśli $A$ ma element największy, to jest on również supremum tego zbioru.
    \end{theorem}
    \begin{proof}
        Z definicji majoranty wynika, że element największy zbioru $A$ jest również jego majorantą. Każda majoranta $M \in X$ zbioru $A$ oczywiście jest ,,większa'' niż dowolny element zbioru $A$ (w tym również jego element największy $\ol{M}$), to znaczy
        $$ \forall M : \ol{M} \preceq M, $$
        z czego wynika, że $\ol{M}$ jest elementem najmniejszym zbioru majorant zbioru $A$, a więc supremum tego zbioru.
    \end{proof}

    \begin{corollary}
        \label{c:if_sup_out_then_no_greatest}
        Jeśli zbiór częściowo uporządkowany $X$ ma supremum, które nie należy do tego zbioru, to zbiór $X$ nie ma elementu największego.
    \end{corollary}
    \begin{proof}
        Ponieważ dowolny zbiór (na mocy twierdzenia \ref{t:uniq_greatest}) ma co najwyżej jedno supremum, to gdyby zbiór $X$ miał element najwiekszy, to na mocy twierdzenia \ref{t:greatest=sup} byłoby ono również supremum, które należy do zbioru $X$.
    \end{proof}

    \begin{example}
        Weźmy zbiór liniowo uporządkowany $(\RR, \leq)$ oraz jego podzbiór $A = [0, 1) \subset \RR$. Zbiór majorant zbioru $A$ to przedział $[1, \infty)$, a jego najmniejszy element (a zarazem supremum zbioru $A$) to liczba $1$. Mamy więc
        $$ \sup A = 1. $$
        Liczba $1$ nie należy jednak do zbioru $A$, więc, na mocy wniosku \ref{c:if_sup_out_then_no_greatest}, element największy (a z faktu \ref{f:greatest=maximal} również maksymalny) nie istnieje.
    \end{example}

    \begin{example}
        Weźmy zbiór częściowo uporządkowany $(\CC, \preceq)$, gdzie zdefiniujemy
        $$ x \preceq y \iff \Re{x} \leq \Re{y} \wedge \Im{x} \leq \Im{y}. $$
        Oczywiście niektóre elementy nie będą w tym porządku porównywalne, na przykład $1$ oraz $i$.

        Weźmy również podzbiór $A \subset \CC$ taki, że
        $$ A = \{z : |z| \leq 1\}. $$

        Na rysunku zaznaczono \textcolor{MainColor1}{zbiór $A$}, \textcolor{LinkColor1}{zbiór majorant $M$ zbioru $A$}, \textcolor{BoxColor1}{supremum zbioru~$A$} oraz \textcolor{MainColor1}{zbiór elementów maksymalnych} (jako ćwierćokrąg). Na mocy wniosku \ref{c:if_sup_out_then_no_greatest} element największy nie istnieje.

        \begin{center}
            \begin{tikzpicture}
                \tkzInit[xmin=-1.1, xmax=3.7, ymin=-1.1, ymax=2.7]
                \tkzDefPoints{0/0/O,1/0/A,0/1/B}
                \tkzDefPoints{1/1/M_1,4/1/M_2,4/3/M_3,1/3/M_4}
                \tkzGrid
                \tkzDrawX[label=$\Re$,thick] \tkzDrawY[label=$\Im$,thick]
                \tkzClip
                \tkzDrawCircle[color=MainColor1, line width=2pt, fill=MainColor1!50, opacity=.5](O,A)
                \tkzDrawPolygon[color=LinkColor1, line width=2pt, fill=LinkColor1!50, opacity=.5](M_1,M_2,M_3,M_4)
                \tkzDrawPoint[color=BoxColor1, size=4pt](M_1)
                \tkzDrawArc[color=MainColor1, line width=3pt, opacity=.5](O,A)(B)
            \end{tikzpicture}
        \end{center}
    \end{example}

    \begin{definition}
        Łańcuch to taki podziór $C \subset X$, że $(X, \preceq)$ jest zbiorem z porządkiem częściowym, a $(C, \preceq)$ jest zbiorem z porządkiem liniowym.
    \end{definition}

    \begin{definition}
        Silny porządek to relacja, która jest przechodnia i asymetryczna. Silnie uporządkowany zbiór $X$ oznaczamy przez $(X, \prec)$.
    \end{definition}

    \section{Struktury algebraiczne}
    \vocab{Działaniem} (wewnętrznym) w zbiorze $A$ nazwiemy każde odwzorowanie $h$ takie, że
    $$ h : A \times A \lthen A. $$
    \vocab{Działaniem zewnętrznym} w zbiorze $A$ jest odwzorowanie
    $$ h : F \times A \lthen A. $$

    Jeśli zamiast $h$ weźmiemy jakiś symbol, na przykład $\circ$, to zamiast $h(a, b)$ będziemy pisać $a \circ b$.

    \begin{definition}[rodzaje działań]
        W zbiorze z działaniem $(A, \circ)$ działanie $\circ$ jest:
        \begin{itemize}[--]
            \item \vocab{łączne} $\iff \forall x, y, z \in A : (x \circ y) \circ z = x \circ (y \circ z)$,
            \item \vocab{przemienne} $\iff \forall x, y, \in A : x \circ y = y \circ x$.
        \end{itemize}
    \end{definition}

    Jeśli dla pewnego elementu $e \in A$ zachodzi
    $$ \forall x \in A : x \circ e = e \circ x = x, $$
    to $e$ jest \vocab{elementem neutralnym}.

    \begin{fact}
        Jeżeli w zbiorze $A$ z działaniem $\circ$ istnieje element neutralny, to jest on jedyny.
    \end{fact}
    \begin{proof}
        Jeśli mielibyśmy dwa elementy neutralne $e_1, e_2$ to mamy
        $$ e_1 \circ e_2 = e_1 = e_2. $$
    \end{proof}

    Jeżeli istnieje element neutralny $e \in A$ działania $\circ$, to \vocab{elementem symetrycznym} do $x \in A$ jest taki element $x' \in A$, że
    $$ x \circ x' = e = x' \circ x. $$

    \begin{lemma}
        Jeśli działanie $\circ$ jest łączne w zbiorze $A$ i istnieje element neutralny $e \in A$, to jeśli dany element $x \in A$ ma element symetryczny, to jest on jedyny oraz zachodzi $(x')' = x$.
    \end{lemma}
    \begin{proof}
        Jeśli mielibyśmy dwa elementy symetryczne $x'_1, x'_2$, to mamy
        $$ x'_1 = x'_1 \circ e = x'_1 \circ (x \circ x'_2) = (x'_1 \circ x) \circ x'_2 = e \circ x'_2 = x'_2. $$

        Ponadto z definicji elementu symetrycznego mamy
        $$ x' \circ x = e $$
        oraz
        $$ x' \circ (x')' = e, $$
        a więc $x$ jest elementem symetrycznym $x'$, ergo $(x')' = x$.
    \end{proof}

    \subsection{Grupy}
    \begin{definition}
        Grupa to para $(A, \circ)$, gdzie $A$ jest zbiorem, a działanie $\circ$ jest:
        \begin{enumerate}[noitemsep,nolistsep]
            \item wewnętrzne,
            \item łączne,
            \item ma element neutralny,
            \item a każdy element $x \in A$ ma element symetryczny.
        \end{enumerate}
    \end{definition}

    \begin{definition}
        Grupa abelowa (przemienna) to grupa, w której działanie $\circ$ jest przemienne.
    \end{definition}

    \begin{example}
        Przykłady grup:
        \begin{enumerate}
            \item $(\ZZ, +)$ -- grupa abelowa,
            \item $(\ZZ_n, +_n)$ -- grupa abelowa\footnote{gdzie $\ZZ_n$ oznacza zbiór $\{0, 1, \ldots, n-1\}$, a $+_n$ operację dodawania modulo $n$},
            \item $(\QQ_+, \cdot)$ -- grupa abelowa,
            \item grupą nieabelową jest grupa obrotów danego obiektu o $90\dg$ względem dowolnej z trzech osi.
        \end{enumerate}
    \end{example}

    \begin{theorem}
        \label{t:prime_n->group}
        $(\ZZ_n \setminus \{0\}, \cdot_n)$ jest grupą wtedy i tylko wtedy, gdy $n \geq 2$ jest liczbą pierwszą.
    \end{theorem}
    Łatwo sprawdzić, że mnożenie modulo $n$ w zbiorze $\ZZ_n \setminus \{0\}$ jest wewnętrze i łączne. Ma również element neutralny $1$. Będziemy więc dowodzić jedynie istnienia elementu symetrycznego dla każdego elementu.
    \begin{proof}[Dowód wystarczalności]\renewcommand{\qedsymbol}{}
        Załóżmy przeciwnie, że istnieje $k \in \ZZ_n \setminus \{0, 1\}$ takie, że $k \mid n$. Skoro $(\ZZ_n \setminus \{0\}, \cdot_n)$ jest grupą, to $k$ ma element symetryczny $k^{-1}$. Zachodzi więc
        $$ kk^{-1} \equiv 1 \pmod{n}, $$
        czyli inaczej
        $$ \exists m \in \ZZ : kk^{-1} - 1 = mn. $$
        Co jednak prowadzi do sprzeczności, ponieważ
        $$ kk^{-1} - 1 \not\equiv mn \pmod{k} $$
        $$ - 1 \not\equiv 0 \pmod{k}. $$
    \end{proof}
    \begin{proof}[Dowód dostateczności]
        Skoro $n$ jest liczbą pierwszą, to z małego twierdzenia Fermata mamy
        $$ a^{n-1} \equiv 1 \pmod{n} $$
        dla każdego $a \in \ZZ_n \setminus \{0\}$.
        Z tego wynika, że dla dowolnego elementu $a$ jego elementem symetrycznym będzie $a^{n-2}$.
    \end{proof}

    \subsection{Pierścienie i ciała}
    \begin{definition}
        Pierścień to trójka $(P, \circ, *)$, gdzie $P$ jest zbiorem, $\circ, *$ to działania wewnętrzne oraz
        \begin{enumerate}[noitemsep,nolistsep]
            \item $(P, \circ)$ jest grupą abelową
            \item działanie $*$ jest łączne
            \item działanie $*$ jest rozdzielne względem $\circ$, czyli
            $$\forall x, y, z \in P : \begin{aligned}& (x \circ y) * z = (x * z) \circ (y * z), \\
                                                     & x * (y \circ z) = (x * y) \circ (x * z).\end{aligned} $$
        \end{enumerate}
    \end{definition}

    \begin{definition}
        Pierścień przemienny to pierścień $(P, \circ, *)$, w którym $*$ jest działaniem przemiennym\footnote{wtedy też rozdzielność prawo- i lewostronna stają się tożsame}.
    \end{definition}

    Pierwsze działanie w pierścieniu nazywamy \vocab{działaniem addytywnym} i oznaczamy przez $+$. Element neutralny tego działania nazywamy zerem ($\mathbf{0}$), a element symetryczny do elementu $x$ nazywamy elementem przeciwnym i oznaczamy $-x$.

    Drugie działanie nazywamy \vocab{działaniem multiplikatywnym} i oznaczamy przez $\cdot$. Jeśli w $P$ dodatkowo istnieje element neutralny tego działania, to ten element nazywamy jedynką ($\mathbf{1}$), a pierścień nazywamy \vocab{pierścieniem z jedynką}. Element symetryczny do elementu $x$ nazywamy elementem odwrotnym i oznaczamy $x^{-1}$.

    \begin{definition}
        Dzielnikiem zera jest taki element pierścienia $a \neq \mathbf{0}$, że istnieje niezerowy element $b$, dla którego zachodzi $a \cdot b = \mathbf{0}$.
    \end{definition}

    \begin{definition}
        Pierścień całkowity to pierścień z jedynką, w którym nie ma dzielników zera.
    \end{definition}

    \begin{lemma}
        \label{l:cancellation_property}
        W pierścieniach całkowitych zachodzi \vocab{własność skracania}, to znaczy, że dla elementów pierścienia $a, b, c$ przy $c \neq \mathbf{0}$ zachodzi
        $$ ac = bc \implies a = b. $$
    \end{lemma}
    \begin{proof}
        Jeśli $ac = bc$, to $ac - bc = \mathbf{0}$. Z rozdzielności dostajemy
        $$ (a - b)c = \mathbf{0}. $$
        W pierścieniu całkowitym nie ma jednak dzielników zera, więc $a - b = \mathbf{0}$, co dowodzi tezy.
    \end{proof}

    \begin{definition}
        Ciało to pierścień z jedynką, w którym dla każdego elementu $x \neq \mathbf{0}$ istnieje element odwrotny $x^{-1}$.
    \end{definition}

    Ciałem przemiennym będzie ciało, w którym działanie $\cdot$ jest przemienne. Niektórzy autorzy utożsamiają pojęcie ciała z ciałem przemiennym.

    Można zauważyć, że struktura $(K, +, \cdot)$ jest ciałem (przemiennym) jeżeli:
    \begin{enumerate}[noitemsep,nolistsep]
        \item $(K, +)$ jest grupą abelową,
        \item $(K \setminus \{\mathbf{0}\}, \cdot)$ jest grupą (przemienną),
        \item zachodzi warunek rozdzielności $\cdot$ względem $+$.
    \end{enumerate}

    \begin{lemma}
        \label{l:a0=0}
        Dla każdego elementu ciała $a$ zachodzi $a \cdot \mathbf{0} = \mathbf{0}$.
    \end{lemma}
    \begin{proof}
        \begin{align*}             a \cdot \mathbf{0} &= a \cdot (\mathbf{0} + \mathbf{0}) \\
                                   a \cdot \mathbf{0} &= a \cdot \mathbf{0} + a \cdot \mathbf{0} \\
            a \cdot \mathbf{0} + - a \cdot \mathbf{0} &= a \cdot \mathbf{0} + a \cdot \mathbf{0} + - a \cdot \mathbf{0} \\
                                           \mathbf{0} &= a \cdot \mathbf{0} + \mathbf{0} \\
                                           \mathbf{0} &= a \cdot \mathbf{0}
        \end{align*}
    \end{proof}

    \begin{theorem}
        Każde ciało jest pierścieniem całkowitym.
    \end{theorem}
    \begin{proof}
        Załóżmy przeciwnie, że istnieją dzielniki zera, czyli takie dwa elementy ciała $x, y$, że $x, y \neq \mathbf{0}$ oraz $x \cdot y = \mathbf{0}$. Mamy
        \begin{align*}
            x \cdot y &= \mathbf{0} \\
            x^{-1} \cdot x \cdot y &= x^{-1} \cdot \mathbf{0} \\
            y &= x^{-1} \cdot \mathbf{0},
        \end{align*}
        co, na mocy lematu \ref{l:a0=0}, jest sprzecznością z założeniem.
    \end{proof}

    \begin{theorem}
        Każdy skończony pierścień całkowity jest ciałem.
    \end{theorem}
    \begin{proof}
        Załóżmy przeciwnie, że istnieje element pierścienia $a \neq \mathbf{0}$, który nie ma elementu odwrotnego. Rozważmy iloczyny $aa_1, aa_2, aa_3, \ldots$ elementu $a$ ze wszystkimi innymi elementami pierścienia (w tym z $\mathbf{1}$). Z założenia nie ma wsród nich jedynki, więc, skoro $\cdot$ jest działaniem wewnętrznym, to z zasady szufladkowej istnieją takie $a_k \neq a_l$, że $aa_k = aa_l$. To stwierdzenie jest jednak sprzecznością na mocy lematu \ref{l:cancellation_property}, ponieważ rozważamy pierścienie całkowite, w których nie ma dzielników zera.
    \end{proof}

    \begin{example}
        Przykłady pierścieni i ciał:
        \begin{itemize}
            \item $(\ZZ, +, \cdot)$ -- pierścień całkowity, który nie jest ciałem (nie ma dzielników zera, ale często elementy odwrotne nie zawierają się z zbiorze $\ZZ$),
            \item $(\QQ, +, \cdot)$ -- ciało liczb wymiernych,
            \item $(\RR, +, \cdot)$ -- ciało liczb rzeczywistych,
            \item $(\CC, +, \cdot)$ -- ciało liczb zespolonych,
            \item $(\ZZ_n, +_n, \cdot_n)$ -- pierścień przemienny z jedynką.
        \end{itemize}
    \end{example}

    \begin{corollary}[z twierdzenia \ref{t:prime_n->group}]
        Pierścień $(\ZZ_n, +_n, \cdot_n)$ jest ciałem wtedy i tylko wtedy, gdy $n$ jest liczbą pierwszą.
    \end{corollary}

    \subsection{Morfizmy}
    \begin{definition}
        \label{d:homomorphism}
        Homomorfizmem grupy $(A_1, +)$ w grupę $(A_2, \oplus)$ jest takie odwzorowanie $h : A_1 \lthen A_2$, że
        $$ \forall x, y \in A_1 : h(x + y) = h(x) \oplus h(y). $$
    \end{definition}

    \begin{fact}
        Jeśli $h : A_1 \lthen A_2$ jest homomorfizmem grupy $(A_1, +)$ w $(A_2, \oplus)$, to
        \begin{enumerate}[noitemsep,nolistsep]
            \item $e \in A_1$ jest elementem neutralnym w $(A_1, +)$ $\Longrightarrow$ $h(e) \in A_2$ jest elementem neutralnym w $(A_2, \oplus)$,
            \item $\forall x \in A_1 : h(x') = h(x)'$.
        \end{enumerate}
    \end{fact}

    \begin{definition}
        Izomorfizm między grupami $(A_1, +), (A_2, \oplus)$ jest homomorfizmem bijektywnym. Jeśli taki izomorfizm istnieje, to dwie grupy nazywamy izomorficznymi.
    \end{definition}

    \begin{definition}
        Automorfizm to izomorfizm struktury na samą siebie.
    \end{definition}

    Analogicznie definiujemy morfizmy między pierścieniami i ciałami (wtedy równość z definicji \ref{d:homomorphism} musi zachodzić dla obydwu działań).

    \begin{example}
        Przykłady morfizmów:
        \begin{itemize}
            \item $h(x) = x^2$ jest homomorfizmem grupy $(\RR \setminus \{0\}, \cdot)$ w $(\RR_+, \cdot)$,
            \item $h(x) = e^x$ jest izomorfizmem grupy $(\RR, +)$ w $(\RR_+, \cdot)$, ponieważ
            $$ h(x + y) = e^{x + y} = e^x \cdot e^y = h(x) \cdot g(y), $$
            \item $h(z) = \ol{z}$ jest automorfizmem grupy $(\CC, +).$
        \end{itemize}
    \end{example}

    Na podobnej zasadzie jak w przykładzie drugim, można pokazać izomorfizm grupy $(\ZZ_n, +_n)$ z grupą pierwiastów $n$-tego stopnia z jedności względem mnożenia $(\mu_n(\CC), \cdot)$. Biorąc funkcję $h(x) = \cos(\frac{2\pi}{n}x) + i\sin(\frac{2\pi}{n}x)$, mamy
    \begin{align*} h(x + y) &= \cos(\tfrac{2\pi}{n}(x + y)) + i\sin(\tfrac{2\pi}{n}(x + y)) \\
        &= \left(\cos(\tfrac{2\pi}{n}x) + i\sin(\tfrac{2\pi}{n}x)\right) \cdot \left(\cos(\tfrac{2\pi}{n}y) + i\sin(\tfrac{2\pi}{n}y)\right) = h(x) \cdot h(y)\end{align*}

\end{document}